\documentclass[10pt,a4paper]{article}
\usepackage[utf8]{inputenc}
\usepackage[margin=1.8cm]{geometry}
\usepackage[T1]{fontenc}
\usepackage[french]{babel}
\usepackage{amsmath}
\usepackage{amssymb}
\usepackage{tcolorbox}
\usepackage{amsfonts}
\usepackage{stmaryrd}
\usepackage{xcolor}
\title{Projet Programmation 2 : Rapport Partie 2}
\author{MANGEL Léo et MICHEL Thomas}
\date{}
\begin{document}
\maketitle

\section{Fonctionnalités ajoutés}

\subsection{Curseur, armes à distance et objets lançable}

Afin de pouvoir viser de manière précise, nous avons ajouté un curseur qui devient visible lorsqu'on tire avec une arme ou lorsqu'on lance un objet. 

\subsection{Sauvegarde}

Pour sauvegarder les informations du jeu en cours, nous utilisons la sérialisation native de Scala. Cela nous permet de rendre chaque classe sauvegardable dans un fichier en ajoutant le trait \texttt{Serializable}. Ainsi, lorsqu'on sauvegarde, on enregistre dans un fichier un objet de la classe \texttt{Game} qui contient toutes les informations sur la partie en cours. De même, lorsqu'on charge, on désérialise le contenu du fichier pour obtenir l'objet de la classe \texttt{Game} initial.\\
Pour sauvegarder et charger une partie, il y a un menu principal, auquel on a accès en début de jeu, ainsi qu'à n'importe quel moment de la partie en appuyant sur Echap. Nous avons choisi d'avoir une seule sauvegarde possible, afin de ne pas avoir besoin d'implémenter un système pour sélectionner une sauvegarde.\\

\subsection{Niveaux et objectifs}

\section{Difficultés rencontrées}

Malgré sa grande simplicité de mise en place dans le code, nous avons eu des difficultés à faire fonctionner correctement la sérialisation. En effet, sbt empêche initialement au programme de faire des forks, ce qui est utilisé par Scala pour la désérialisation. Pour régler le problème, il faut changer le fichier \texttt{build.sbt}. Nous avons pensé un peu de temps à comprendre d'ou venait le problème et à le résoudre.\\



\end{document}

